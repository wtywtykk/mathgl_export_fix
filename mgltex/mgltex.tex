\documentclass[12pt]{article}

\usepackage[png]{mgltex}

\title{\mglTeX{} package example}
\author{Diego Sejas Viscarra, Alexey Balakin}
\date{\today}
\mgldir{scripts/}

\begin{document}
  
\maketitle


LaTeX package \texttt{mgltex} (was made by Diego Sejas Viscarra) allow one to make figures directly from MGL script located in LaTeX file.

For using this package you need to specify \texttt{--shell-escape} option for \emph{latex/pdflatex} or manually run \emph{mglconv} tool with produced MGL scripts for generation of images. Don't forgot to run \emph{latex/pdflatex} second time to insert generated images into the output document.

The package may have following options: \texttt{draft}, \texttt{final} -- the same as in the \emph{graphicx} package; \texttt{jpg}, \texttt{jpeg}, \texttt{png} -- for export to JPEG/PNG images; \texttt{eps}, \texttt{epsz} -- for export to uncompressed/compressed EPS format as primitives; \texttt{bps}, \texttt{bpsz} -- for export to uncompressed/compressed EPS format as bitmap (don't work with \emph{pdflatex}), \texttt{pdf} -- for export to 3D PDF.

The package defines the following environments:
\begin{description}
\item[mgl]
	It writes its contents to a general script which has the same name as the LaTeX document, but its extension is \emph{.mgl}. The code in this environment is compiled and the image produced is included. It takes exactly the same optional arguments as the \texttt{\textbackslash{}includegraphics} command, plus an additional argument \emph{imgext}, which specifies the extension to save the image.
\item[mgladdon]
	It adds its contents to the general script, without producing any image. It useful to set some global properties (like size of the images) at beginning of the document.
\item[mglcode]
	Is exactly the same as \texttt{mgl}, but it writes its contents verbatim to its own file, whose name is specified as a mandatory argument.
\item[mglscript]
	Is exactly the same as \texttt{mglcode}, but it doesn't produce any image, nor accepts optional arguments. It is useful, for example, to create a MGL script, which can later be post processed by another package like "listings".
\item[mglblock]
	It writes its contents verbatim to a file, specified as a mandatory argument, and to the LaTeX document, and numerates each line of code.

% This last three environments will test if the user is overwriting some file, and will issue a warning in that case.
\item[mglverbatim]
	Exactly the same as \texttt{mglblock}, but it doesn't write to a file. This environment doesn't have arguments.
\item[mglfunc]
	Is used to define MGL functions. It takes one mandatory argument, which is the name of the function, plus one additional argument, which specifies the number of arguments of the function. The environment needs to contain only the body of the function, since the first and last lines are appended automatically, and the resulting code is written at the end of the general script, which is also written automatically. The warning is produced if 2 or more function with the same name is defined.
\item[mglsetup]
	If many scripts with the same code are to be written, the repetitive code can be written inside this environment only once, then this code will be used automatically every time the \texttt{\textbackslash{}mglplot} command is used (see below). It takes one optional argument, which is a name to be associated to the corresponding contents of the environment; this name can be passed to the \texttt{\textbackslash{}mglplot} command to use the corresponding block of code automatically (see below).
\end{description}

The package also defines the following commands:
\begin{description}
\item[\textbackslash{}mglplot]
	It takes one mandatory argument, which is MGL instructions separated by the symbol ':' this argument can be more than one line long. It takes the same optional arguments as the \texttt{mgl} environment, plus an additional argument \emph{setup}, which indicates the name associated to a block of code inside a \texttt{mglsetup} environment. The code inside the mandatory argument will be appended to the block of code specified, and the resulting code will be written to the general script.
\item[\textbackslash{}mglgraphics]
	This command takes the same optional arguments as the \texttt{mgl} environment, and one mandatory argument, which is the name of a MGL script. This command will compile the corresponding script and include the resulting image. It is useful when you have a script outside the LaTeX document, and you want to include the image, but you don't want to type the script again.
\item[\textbackslash{}mglinclude]
	This is like \texttt{\textbackslash{}mglgraphics} but, instead of creating/including the corresponding image, it writes the contents of the MGL script to the LaTeX document, and numerates the lines.
\item[\textbackslash{}mgldir]
	This command can be used in the preamble of the document to specify a directory where LaTeX will save the MGL scripts and generate the corresponding images. This directory is also where \texttt{\textbackslash{}mglgraphics} and \texttt{\textbackslash{}mglinclude} will look for scripts.
\item[\textbackslash{}mglTeX]
	It just pretty prints the name of the package ''\mglTeX''.
\end{description}


An example of usage of \texttt{mgl} and \texttt{mglfunc} environments would be:
\begin{verbatim}
\begin{mglfunc}{prepare2d}
  new a 50 40 '0.6*sin(pi*(x+1))*sin(1.5*pi*(y+1))+0.4*cos(0.75*pi*(x+1)*(y+1))'
  new b 50 40 '0.6*cos(pi*(x+1))*cos(1.5*pi*(y+1))+0.4*cos(0.75*pi*(x+1)*(y+1))'
\end{mglfunc}

\begin{figure}[!ht]
  \centering
  \begin{mgl}[width=0.85\textwidth,height=7.5cm]
    fog 0.5
    call 'prepare2d'
    subplot 2 2 0:title 'Surf plot (default)':rotate 50 60:light on:box:surf a

    subplot 2 2 1:title '"\#" style; meshnum 10':rotate 50 60:box
    surf a '#'; meshnum 10

    subplot 2 2 2 : title 'Mesh plot' : rotate 50 60 : box
    mesh a

    new x 50 40 '0.8*sin(pi*x)*sin(pi*(y+1)/2)'
    new y 50 40 '0.8*cos(pi*x)*sin(pi*(y+1)/2)'
    new z 50 40 '0.8*cos(pi*(y+1)/2)'
    subplot 2 2 3 : title 'parametric form' : rotate 50 60 : box
    surf x y z 'BbwrR'
  \end{mgl}
\end{figure}
\end{verbatim}
Note, that \texttt{mglfunc} environment(s) can be located at any position (at the beginning, at the end, or somewhere else) of LaTeX document.
\begin{mglfunc}{prepare2d}
  new a 50 40 '0.6*sin(pi*(x+1))*sin(1.5*pi*(y+1))+0.4*cos(0.75*pi*(x+1)*(y+1))'
  new b 50 40 '0.6*cos(pi*(x+1))*cos(1.5*pi*(y+1))+0.4*cos(0.75*pi*(x+1)*(y+1))'
\end{mglfunc}

\begin{figure}[!ht]
  \centering
  \begin{mgl}[width=0.85\textwidth,height=7.5cm]
    fog 0.5
    call 'prepare2d'
    subplot 2 2 0 : title 'Surf plot (default)' : rotate 50 60 : light on : box : surf a

    subplot 2 2 1 : title '"\#" style; meshnum 10' : rotate 50 60 : box
    surf a '#'; meshnum 10

    subplot 2 2 2 : title 'Mesh plot' : rotate 50 60 : box
    mesh a

    new x 50 40 '0.8*sin(pi*x)*sin(pi*(y+1)/2)'
    new y 50 40 '0.8*cos(pi*x)*sin(pi*(y+1)/2)'
    new z 50 40 '0.8*cos(pi*(y+1)/2)'
    subplot 2 2 3 : title 'parametric form' : rotate 50 60 : box
    surf x y z 'BbwrR'
  \end{mgl}
\end{figure}

Following example show the usage of \texttt{mglscript} environment
\begin{verbatim}
\begin{mglscript}{Vectorial}
call 'prepare2v'
subplot 3 2 0 '' : title 'lolo' : box
vect a b
subplot 3 2 1 '' : title '"." style; "=" style' : box
vect a b '.='
subplot 3 2 2 '' : title '"f" style' : box
vect a b 'f'
subplot 3 2 3 '' : title '">" style' : box
vect a b '>'
subplot 3 2 4 '' : title '"<" style' : box
vect a b '<'
call 'prepare3v'
subplot 3 2 5 : title '3d variant' : rotate 50 60 : box
vect ex ey ez

stop
    
func 'prepare2v'
  new a 20 30 '0.6*sin(pi*(x+1))*sin(1.5*pi*(y+1))+0.4*cos(0.75*pi*(x+1)*(y+1))'
  new b 20 30 '0.6*cos(pi*(x+1))*cos(1.5*pi*(y+1))+0.4*cos(0.75*pi*(x+1)*(y+1))'
return
    
func 'prepare3v'
  define $1 pow(x*x+y*y+(z-0.3)*(z-0.3)+0.03,1.5)
  define $2 pow(x*x+y*y+(z+0.3)*(z+0.3)+0.03,1.5)
  new ex 10 10 10 '0.2*x/$1-0.2*x/$2'
  new ey 10 10 10 '0.2*y/$1-0.2*y/$2'
  new ez 10 10 10 '0.2*(z-0.3)/$1-0.2*(z+0.3)/$2'
return
\end{mglscript}
\end{verbatim}

\begin{mglscript}{Vectorial}
call 'prepare2v'
subplot 3 2 0 '' : title 'lolo' : box
vect a b
subplot 3 2 1 '' : title '"." style; "=" style' : box
vect a b '.='
subplot 3 2 2 '' : title '"f" style' : box
vect a b 'f'
subplot 3 2 3 '' : title '">" style' : box
vect a b '>'
subplot 3 2 4 '' : title '"<" style' : box
vect a b '<'
call 'prepare3v'
subplot 3 2 5 : title '3d variant' : rotate 50 60 : box
vect ex ey ez

stop
    
func 'prepare2v'
  new a 20 30 '0.6*sin(pi*(x+1))*sin(1.5*pi*(y+1))+0.4*cos(0.75*pi*(x+1)*(y+1))'
  new b 20 30 '0.6*cos(pi*(x+1))*cos(1.5*pi*(y+1))+0.4*cos(0.75*pi*(x+1)*(y+1))'
return
    
func 'prepare3v'
  define $1 pow(x*x+y*y+(z-0.3)*(z-0.3)+0.03,1.5)
  define $2 pow(x*x+y*y+(z+0.3)*(z+0.3)+0.03,1.5)
  new ex 10 10 10 '0.2*x/$1-0.2*x/$2'
  new ey 10 10 10 '0.2*y/$1-0.2*y/$2'
  new ez 10 10 10 '0.2*(z-0.3)/$1-0.2*(z+0.3)/$2'
return
\end{mglscript}

You should use \texttt{\textbackslash{}mglgraphics} command to display its contents
\begin{verbatim}
\begin{figure}[!ht]
  \centering
  \mglgraphics[width=40em,height=20em]{Vectorial}
  \caption{A beautiful example}
\end{figure}
\end{verbatim}

\begin{figure}[!ht]
  \centering
  \mglgraphics[width=40em,height=20em]{Vectorial}
  \caption{A beautiful example}
\end{figure}

Alternatively, you can display the contents of the script in parallel to saving to a file, if you are using \texttt{mglblock} environment
\begin{verbatim}
\begin{mglblock}{Axis_projection}
  ranges 0 1 0 1 0 1
  new x 50 '0.25*(1+cos(2*pi*x))'
  new y 50 '0.25*(1+sin(2*pi*x))'
  new z 50 'x'
  new a 20 30 '30*x*y*(1-x-y)^2*(x+y<1)'
  new rx 10 'rnd':new ry 10:fill ry '(1-v)*rnd' rx
  light on
  
  title 'Projection sample':ternary 4:rotate 50 60
  box:axis:grid
  plot x y z 'r2':surf a '#'
  xlabel 'X':ylabel 'Y':zlabel 'Z'
\end{mglblock}
\begin{figure}[!ht]
  \centering
  \mglgraphics[scale=0.5]{Axis_projection}
  \caption{The image from Axis\_projection.mgl script}
\end{figure}
\end{verbatim}

\begin{mglblock}{Axis_projection}
  ranges 0 1 0 1 0 1
  new x 50 '0.25*(1+cos(2*pi*x))'
  new y 50 '0.25*(1+sin(2*pi*x))'
  new z 50 'x'
  new a 20 30 '30*x*y*(1-x-y)^2*(x+y<1)'
  new rx 10 'rnd':new ry 10:fill ry '(1-v)*rnd' rx
  light on
  
  title 'Projection sample':ternary 4:rotate 50 60
  box:axis:grid
  plot x y z 'r2':surf a '#'
  xlabel 'X':ylabel 'Y':zlabel 'Z'
\end{mglblock}
\begin{figure}[!ht]
  \centering
  \mglgraphics[scale=0.5]{Axis_projection}
  \caption{The image from Axis\_projection.mgl script}
\end{figure}

Finally, you can just show MGL script itself
\begin{verbatim}
\begin{mglverbatim}
  ranges 0 1 0 1 0 1
  new x 50 '0.25*(1+cos(2*pi*x))'
  new y 50 '0.25*(1+sin(2*pi*x))'
  new z 50 'x'
  new a 20 30 '30*x*y*(1-x-y)^2*(x+y<1)'
  new rx 10 'rnd':new ry 10:fill ry '(1-v)*rnd' rx
  light on
  
  title 'Projection sample':ternary 4:rotate 50 60
  box:axis:grid
  plot x y z 'r2':surf a '#'
  xlabel 'X':ylabel 'Y':zlabel 'Z'
\end{mglverbatim}
\end{verbatim}

\begin{mglverbatim}
  ranges 0 1 0 1 0 1
  new x 50 '0.25*(1+cos(2*pi*x))'
  new y 50 '0.25*(1+sin(2*pi*x))'
  new z 50 'x'
  new a 20 30 '30*x*y*(1-x-y)^2*(x+y<1)'
  new rx 10 'rnd':new ry 10:fill ry '(1-v)*rnd' rx
  light on
  
  title 'Projection sample':ternary 4:rotate 50 60
  box:axis:grid
  plot x y z 'r2':surf a '#'
  xlabel 'X':ylabel 'Y':zlabel 'Z'
\end{mglverbatim}


An example of usage of \texttt{\textbackslash{}mglplot} command would be:
\begin{verbatim}
\begin{mglsetup}
  box '@{W9}' : axis
\end{mglsetup}
\begin{mglsetup}[2d]
  box : axis
  grid 'xy' ';k'
\end{mglsetup}
\begin{mglsetup}[3d]
  rotate 50 60
  box : axis : grid 'xyz' ';k'
\end{mglsetup}
\begin{figure}[!ht]
  \centering
  \mglplot[scale=0.5]{new a 200 'sin(pi*x)':plot a '2B'}
\end{figure}
\begin{figure}[!ht]
  \centering
  \mglplot[scale=0.5,setup=2d]{
    fplot 'sin(pi*x)' '2B' :
    fplot 'cos(pi*x^2)' '2R'
  }
\end{figure}
\begin{figure}[!ht]
  \centering
  \mglplot[width=0.5 \textwidth, setup=3d]
  {fsurf 'sin(pi*x)+cos(pi*y)'}
\end{figure}
\end{verbatim}

\begin{mglsetup}
  box '@{W9}' : axis
\end{mglsetup}
\begin{mglsetup}[2d]
  box : axis
  grid 'xy' ';k'
\end{mglsetup}
\begin{mglsetup}[3d]
  rotate 50 60
  box : axis : grid 'xyz' ';k'
\end{mglsetup}
\begin{figure}[!ht]
  \centering
  \mglplot[scale=0.5]{new a 200 'sin(pi*x)':plot a '2B'}
\end{figure}
\begin{figure}[!ht]
  \centering
  \mglplot[scale=0.5,setup=2d]{
    fplot 'sin(pi*x)' '2B' :
    fplot 'cos(pi*x^2)' '2R'
  }
\end{figure}
\begin{figure}[!ht]
  \centering
  \mglplot[width=0.5 \textwidth, setup=3d]
  {fsurf 'sin(pi*x)+cos(pi*y)'}
\end{figure}

As an additional feature, when an image is not found or cannot be included, instead of issuing an error, \texttt{mgltex} prints a box with the word \emph{'MGL image not found'} in the LaTeX document.
\begin{figure}[!ht]
  \centering
  \mglgraphics{xyz}
\end{figure}

The last sample is displaying the content of the MGL file using \texttt{\textbackslash{}mglinclude} command:
\mglinclude{Vectorial}

\end{document}